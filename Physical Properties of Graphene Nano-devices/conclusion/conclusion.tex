%\documentclass[12pt,a4paper]{report}
%\usepackage[utf8]{inputenc}
%\usepackage{amsmath}
%\usepackage{amsfonts}
%\usepackage{amssymb}
%\usepackage[margin=2.5cm]{geometry}
%\usepackage{graphicx}
%\usepackage{caption}
%\usepackage{subcaption}
%\usepackage[nottoc,numbib]{tocbibind}
%\linespread{1.3}
%\begin{document}
	\chapter{Conclusion}
	\label{Conclusion}

		In this doctoral thesis the two dimensional material graphene has been studied with particular interest in the IV characteristics of Zener tunnelling graphene nano-devices. The work here derives the electronic properties of graphene. Starting from the two dimensional, hexagonal structure of graphene the full energy spectrum for electrons and holes was obtained. The Dirac cones, located in the corners of the Brillouin zone at the Fermi level were then studied in further detail. A two-variable Taylor expansion around each Dirac point provides a Dirac-like Hamiltonian for electrons and holes close to a Dirac point in an infinite sheet of graphene. The energy spectrum around these Dirac points is then used to obtain the density of states. With the density of states and the Landauer formalism an expression for the current and conductance through a graphene scattering device was derived.

		The scattering properties of various graphene devices was then studied. The graphene potential step is a two-region problem representing a p-n junction. The electron-hole interface requires special attention to the velocity of charge carriers and a phase change must be introduced. By considering the conservation of probability current and the direction of charge carriers, the scattering properties through a graphene potential step were obtained. The transmission properties were then discussed with respect to various magnitudes of potentials and energy gaps. This revealed many interesting properties for graphene nano devices. There is perfect transmission through the step when the incident angle of the charge carrier is zero, or if the potential step is sufficiently large. This property is known as Klein tunnelling. The Landauer formalism was then used to determine the current and conductance through such devices for comparison with experimental data.

		Introducing a third region to the potential step creates a graphene potential barrier. The same analysis was applied to the graphene potential barrier; however, as there is an additional electron-hole interface new properties can be observed. A single graphene potential barrier can act as a Fabry-P\'{e}rot resonator. These resonances cause the potential barrier to become transparent. The second electron-hole interface also creates bound states within the potential. Using a system where the wave-functions decay away from the potential the dispersion relation of these bound states can found.

		If the initial region of the graphene potential barrier varies from the third region a combination of the graphene potential step and the graphene potential barrier is created. Depending on the magnitudes of the potentials this creates either a Zener tunnelling barrier, single potential step or a double potential step. The graphene Zener barrier is essentially a potential barrier on top of a potential step which shows properties from both the graphene potential step and the graphene potential barrier. When applying the resonanace condition, or when the potential is large enough for Klein tunnelling the transmission probability becomes that of the graphene potential step present below the barrier. The direction of the potential step will also dramatically alters the transmission properties of the scattering device, possibly creating an additional region of no propagation.

		The theoretical model used here to obtain the current and conductance was then compared to data obtained recently for graphene nano-ribbons. There were striking similarities between the results for infinite sheet graphene and the experimental data for graphene nano-ribbons. By altering the properties of graphene potential structures the theroretical model was used to simulate the experimental data, providing insights into properties of each experimental sample. Possible resolutions to the discrepancies between the theoretical model and experimental data were suggested where appropriate. This includes contact resistance creating a weaker dependence on gate voltage and electronic heating of the experimental samples altering the temperature dependence.

		These methods were then applied to three dimensional materials with a linear dispersion relation. This was done by introducing z directional momentum into the graphene Hamiltonian to produce a system similar to Weyl fermions. The three dimensional scattering properties can be reduced to the graphene case by removing the z momentum from the final expressions and show similar properties to graphene devices. The material Cd$_{3}$As$_{2}$ possesses a three dimensional Dirac cone which is asymmetrical in the z direction. Particular attention was given to the properties of this asymmetry and its effect on the scattering properties in comparison to the symmetrical case. The density of states for three dimensional linear spectrum materials was found to be parabolic with respect to energy. This causes the current and conductance from the Landauer formalism to vary significantly from the graphene case.

		Finally, a WKB analysis of a graphene smooth potential barrier was made. This was achieved by comparing the graphene WKB wave-functions to the wave-functions derived for a double Schr{\" o}dinger barrier. This method produced peaks of perfect transmission and obtained bound states as expected for a graphene nano-device.
%\end{document}
