%\documentclass[12pt,a4paper]{report}
%\usepackage[utf8]{inputenc}
%\usepackage{amsmath}
%\usepackage{amsfonts}
%\usepackage{amssymb}
%\usepackage[margin=2.5cm]{geometry}
%\usepackage{graphicx}
%\usepackage{caption}
%\usepackage{subcaption}
%\usepackage[nottoc,numbib]{tocbibind}
%\linespread{1.3}
%\begin{document}
\thispagestyle{plain}
\begin{center}
	\Large
	\textbf{Novelty}
\end{center}
	In this doctoral thesis the existing analysis into graphene diodes has been expanded to include full current-voltage characterisitics at non-zero temperatures. Using these methods a new device, the Zener barrier, is introduced and its defining properties are explored. The techniques highlighted by graphene research have then been applied to recently discovered topological insulators with three dimensional Dirac cones. The physical properties of diodes and transistors constructed from the new materials have been obtained for the first time. These properties are then modified with a scale factor to accurately simulate the Dirac cones from experimental results.
%\end{document}
