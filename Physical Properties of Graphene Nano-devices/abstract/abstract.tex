%\documentclass[12pt,a4paper]{report}
%\usepackage[utf8]{inputenc}
%\usepackage{amsmath}
%\usepackage{amsfonts}
%\usepackage{amssymb}
%\usepackage[margin=2.5cm]{geometry}
%\usepackage{graphicx}
%\usepackage{caption}
%\usepackage{subcaption}
%\usepackage[nottoc,numbib]{tocbibind}
%\linespread{1.3}
%\begin{document}
\thispagestyle{plain}
\begin{center}
	\Large
	\textbf{Abstract}
\end{center}
	In this doctoral thesis the two dimensional material graphene has been studied in depth with particular respect to Zener tunnelling devices. From the hexagonal structure the Hamiltonian at a Dirac point was derived with the option of including an energy gap. This Hamiltonian was then used to obtain the tunnelling properties of various graphene nano-devices; the devices studied include Zener tunnelling potential barriers such as single and double graphene potential steps. A form of the Landauer formalism was obtained for graphene devices. Combined with the scattering properties of potential barriers the current and conductance was found for a wide range of graphene nano-devices. These results were then compared to recently obtained experimental results for graphene nanoribbons, showing many similarities between nanoribbons and infinite sheet graphene. The methods studied were then applied to materials which have been shown to possess three dimensional Dirac cones known as topological insulators. In the case of Cd$_{3}$As$_{2}$ the Dirac cone is asymmetrical with respect to the $z$ direction, the effect of this asymmetry has been discussed with comparison to the symmetrical case.
%\end{document}
